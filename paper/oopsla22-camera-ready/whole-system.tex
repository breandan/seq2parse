\section{Building a Fast Error-Correcting Parser}
\label{sec:whole-system}

We show how \toolname uses the abstracted token sequences from
\autoref{sec:prog-abstract} and the trained sequence models from
\autoref{sec:seq-classifiers} to generate an \emph{error-correcting parser}
$(e_{\bot} \to e)$, that will parse an input program $e_{\bot}$ with syntax
errors and produce a correct program $e$. We first describe how we extract a
machine-learning-amenable training set from a corpus of fixed programs and
finally how we structure everything to train our model.


\subsection{Learning Error Production Rules}
\label{sec:whole-system:error-rules}

The $\trainDLsym$ method requires a dataset of token sequences $t^a$ that is
annotated with an \emph{exact and small set} of error production rules, \ie
$\List{t^a \times \errorrulessym}$. These $\errorrulessym$ are just a subset of
all the possible error rules that are needed to parse and fix $t^a$. The
straight-forward approach is to use $\ecepsym$ with all possible error
production rules for each program $e_{\bot}$ in the dataset. Then, when
$\ecepsym$ returns with a successful parse, we extract the rules that where used
to parse the program $e_{\bot}$. This approach generates a dataset with the
smallest possible set of error rules as labels per program, since the original
ECE-Parser returns the minimum-distance edit parse. However, this approach
completely ignores the programmer's fix and takes an unreasonable amount of time
to parse a dataset with millions of programs, due to the inefficient nature of
the ECE-Parser.

We suggest using an $O(ND)$ difference algorithm \citep{Myers_1986} to get a
small but still representative set of error production rules for each program
$e_{\bot}$. We employ this algorithm to find the differences between the input
\emph{program token sequence} $t^i$, which is the lexed program $e_{\bot}$ and
the \emph{fixed token sequence} $t^o$, which is the lexed program $e$. This
algorithm returns changes between token sequences in the form of \emph{inserted
or deleted tokens}. It is possible that this algorithm returns a sequence of
deletions followed by a sequence of insertions, which can in turn be interpreted
as a \emph{replacement} of tokens. We map these three types of changes to the
respective error production rules. Let $t^i$ be a sequence $t^i_1, t^i_2, \dots,
t^i_n$ and $t^o$ be the updated sequence $t^o_1, t^o_2, \dots, t^o_m$. We map:
\begin{itemize}
    \item an inserted output token $t^o_j$ to a \emph{deletion} error $E_{t^o_j}
    \rightarrow \epsilon$.
    \item a deleted input token $t^i_k$ to an \emph{insertion} error $Tok\
    \rightarrow t^i_k$ and the helper rule $E_{t^i_{k+1}} \rightarrow Ins\
    t^i_{k+1}$.
    \item a replaced token $t^i_k$ with $t^o_j$ to a \emph{replacement} error
    $Repl\ \rightarrow t^i_k$ and the helper rule $E_{t^o_j} \rightarrow Repl$.
\end{itemize}

In the case of an insertion error, we also include the helper rules $Ins\
\rightarrow\ Tok$ and $Ins\ \rightarrow\ Ins\ Tok$, that can derive any nonempty
sequence of insertions. To introduce (possible) insertion errors at the end of a
program, we include the starting production rules $S' \rightarrow S$ and $S'
\rightarrow S\ Ins$.

The above algorithm, so far, adds only the \emph{terminal error productions}. We
have to include the \emph{non-terminal error productions} that will invoke the
terminal ones. If $X \rightarrow a_0b_0a_1b_1 \dots a_mb_m,\ m \geq 0$, is a
production in $P$ such that $a_i$ is in $N^*$ and $b_i$ is in $\Sigma$, then we
add the error production $X \rightarrow a_0X_{b_0}a_1X_{b_1} \dots a_mX_{b_m},\
m \geq 0$ to $P'$, where each $X_{b_i}$, is either a new non-terminal $E_{b_i}$
that was added with the previous algorithm, or just $b_i$ again if it was not
added.

Finally, we further refine the new small set of error productions for each
program $e_{\bot}$ with ECE-Parser, in order to create the final annotated
dataset $\List{t^a \times \errorrulessym}$. The changes that we extracted from
the programmers' fixes might include irrelevant changes to the parse error fix,
\eg code clean-up. Therefore, filtering with the ECE-Parser is still essential
to annotate each program with the appropriate error production rules. We
implement this error-rule-extracting approach in the function $\diffsym$, which
extracts the token differences between an erroneous program $e_{\bot}$ and a
fixed program $e$ and returns the appropriate error production rules.


\subsection{Training and Using a Transformer Classifier}
\label{sec:whole-system:training-classifier}

\begin{figure}[t]
  \centering
  \begin{minipage}[t]{0.49\linewidth}
    \centering
    \begin{algorithm}[t]
    \caption{Training $\systemsym$'s model $\Model$}
    \label{algo:training-classifier-algo}
    \renewcommand{\algorithmicrequire}{\textbf{Input:}}
    \renewcommand{\algorithmicensure}{\textbf{Output:}}
    \begin{algorithmic}[1]
    \Require{Probabilistic Grammar $G$, $\datasetsym\ Ds$}
    \Ensure{Classifier $Model$}
    \Procedure{Train}{$G,\,Ds$}
    \State $D_{ML} \gets \emptyset$
    \ForAll{$\pbad \times \pfix \in Ds$}
      \State $t^a \gets$ \Call{PartialParse}{$G,\,\pbad$}
      \State $rules \gets$ \Call{DiffRules}{$\pbad \times \pfix$}
      \State $D_{ML} \gets D_{ML}\,\cup\ (t_a \times rules)$
    \EndFor
    \State $Model \gets$ \Call{TrainDL}{$D_{ML}$}
    \State \Return{$Model$}
    \EndProcedure
    \end{algorithmic}
\end{algorithm}

  \end{minipage}
  \hspace{0.02\linewidth}
  \begin{minipage}[t]{0.47\linewidth}
    \centering
    \begin{algorithm}[t]
    \caption{Predicting Templates Algorithm}
    \label{algo:predict-algo}
    \renewcommand{\algorithmicrequire}{\textbf{Input:}}
    \renewcommand{\algorithmicensure}{\textbf{Output:}}
    \begin{algorithmic}[1]
    \Require{Feature Extraction Functions $F$, Fix Templates $Ts$, Program Pair Dataset $D$}
    \Ensure{Predictor $Pr$}
    \Procedure{Predict}{$F,\,Ts,\,D$}
    \State $D_{ML} \gets \emptyset$
    \ForAll{$\pbad \times \pfix \in D$}
      \State $d \gets$ \Call{Extract}{$F,\,Ts,\,\pbad \times \pfix$}
      \State $D_{ML} \gets D_{ML}\,\cup$ \Call{InSlice}{$\pbad,\,d$}
    \EndFor
    \State $Models \gets$ \Call{Train}{$D_{ML}$}
    \State $Data \gets \lambda p.$ \Call{InSlice}{$p,$ \Call{Extract}{$F,\,Ts,\,p \times p$}}
    \State $Pr \gets \lambda p.$ \Call{Map}{$\lambda \tilde{p}.$ \Call{Rank}{$Models,\,\tilde{p}[0]$}, $Data(p)$}
    \State \Return{$Pr$}
    \EndProcedure
    \end{algorithmic}
\end{algorithm}

  \end{minipage}
\end{figure}

% \mypara{Training the Transformer Classifier}
%
Given a (probabilistic) grammar $G$ and a dataset $Ds$,
\Cref{algo:training-classifier-algo} extracts a machine-learning appropriate
dataset $D_{ML}$ in order to $\trainsym$ a Transformer classifier $Model$ with
$\trainDLsym$. The classifier $Model$ can then be used to predict error rules
for new erroneous programs $\pbad$.

The dataset $D_{ML}$ starts as an empty set. For each program pair $\pbad \times
\pfix$, we, first, employ $\partialsym$ with the PCFG $G$ and an erroneous
program $\pbad$ to extract the abstracted token sequence $t^a$. Second, we use
the token difference algorithm $\diffsym$ to extract the specific error $rules$
that fix $\pbad$ based on $\pfix$. The abstracted sequence $t^a$ is annotated
with the label $rules$ and is added to $D_{ML}$. The Transformer classifier
$Model$ is trained with $\trainDLsym$ and the newly extracted dataset $D_{ML}$,
which is finally returned by the algorithm. Finally, the $\trainsym$ing
procedure can be performed offline and thus won't affect the performance of the
final program repair.

% \mypara{Predicting Error Rules}
%
Having trained the Transformer classifier $Model$, we can now predict error
rules $Rls$, that will be used by an ECE-Parser, by using the $\predictsym$
procedure defined in \Cref{algo:predict-algo}. $\predictsym$ uses the same input
grammar $G$ to generate an abstracted token sequence $t^a$ for the program $P$
with the $\partialsym$ procedure. Finally, the $\predictDLsym$ procedure
predicts a small set of error production rules $Rls$ for the sequence $t^a$
given the pre-trained $Model$.

\subsection{Generating an Efficient Error-Correcting Parser}
\label{sec:whole-system:building-ecp}

\begin{algorithm}[t]
    \captionsetup{font=small}
    \caption{Generating the final ECEP}
    \label{algo:ecep-algo}
    \renewcommand{\algorithmicrequire}{\textbf{Input:}}
    \renewcommand{\algorithmicensure}{\textbf{Output:}}
    \begin{algorithmic}[1]
    \Require{Grammar $G$, $\datasetsym\ Ds$}
    \Ensure{Error Correcting Parser $Prs$}
    \Procedure{Seq2Parse}{$G,\,Ds$}
    \State $ps \gets$ \Call{Map}{$\lambda.p \to$ \Call{Snd}{$p$}$,\,Ds$}
    \State $PCFG \gets$ \Call{LearnPCFG}{$G,\,ps$}
    \State $Model \gets$ \Call{Train}{$PCFG,\,Ds$}
    \State \textsc{ERulePredictor} $\gets$ \Call{Predict}{$Model,\,PCFG$}
    \State $Prs \gets (\lambda.\pbad \to$ \Call{EarleyECParse}{\Call{ERulePredictor}{$\pbad$}$,\,\pbad$}$)$
    \State \Return{$Prs$}
    \EndProcedure
    \end{algorithmic}
\end{algorithm}


\Cref{algo:ecep-algo} presents our \emph{neurosymbolic} approach, \toolname.
This is the high-level algorithm that combines everything that we described so
far in the last three sections. \toolname first extracts the fixed programs $ps$
from the dataset $Ds$ to learn a probabilistic context-free grammar $PCFG$ for
the input grammar $G$ with $\learnPCFGsym$. It then $\trainsym$s the Transformer
classifier $Model$ to predict error production rules. We define an error rule
predictor, \textsc{ERulePredictor}, using the $\predictsym$ procedure with the
pre-trained $Model$ and grammar $PCFG$. Finally, the algorithm returns the
ECE-Parser $Prs$, which we define as a function that takes as input an erroneous
program $\pbad$ that uses the \textsc{ERulePredictor} to get the set of error
rules needed by $\ecepsym$ to parse and repair it.
