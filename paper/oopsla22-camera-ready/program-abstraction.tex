
\newcommand{\ccount}{\mathit{count}}

\section{Abstracting Programs with Parse Errors}
\label{sec:prog-abstract}

\toolname abstracts programs (with parse errors) into sequences
of \emph{abstract} tokens that are used to train sequence classifiers.
Next, we explain how a traditional Earley parser can be used to extract
\emph{partial} parses using a Probabilistic Context-Free Grammar (PCFG),
to get a higher level of abstraction that preserves more contextual
information than the low-level sequence output by the lexer.

\begin{figure}[t]
\small
\begin{minipage}[t]{0.42\linewidth}
  \lstDeleteShortInline{|}
  \[
  \boxed{
  \begin{array}{lcl}
    G              & \defeq  & (N,\ \Sigma,\ P,\ S) \\
    \pcfgsym       & \defeq  & (N,\ \Sigma,\ P,\ S,\ W) \\
    \midrule
    G'             & \defeq  & (N',\ \Sigma,\ P',\ S') \\
    P'             & \defeq  & P\ \cup\ \errorrulessym \\
    N'             & \defeq  & N\ \cup\ \{S',\ H,\ I\}\ \cup\\
                   &         & \{E_a\ |\ a \in \Sigma\} \\
    \midrule
    e              & \in    & L(G) \\
    e_{\bot}       & \notin & L(G) \\
    \pairsym       & \defeq & e_{\bot} \times e \\
    \datasetsym    & \defeq & \List{\pairsym} \\
  \end{array}
  }
  \]
  \lstMakeShortInline[language=Python, mathescape=true]{|}
\end{minipage}
\begin{minipage}[t]{0.57\linewidth}
  \lstDeleteShortInline{|}
  \[
  \boxed{
  \begin{array}{lcl}
    \learnPCFGsym  & : & G \to \List{e} \to \pcfgsym \\
    \partialsym    & : & \pcfgsym \to e_{\bot} \to t^a  \\
    \midrule
    \trainDLsym    & : & \List{t^a \times \errorrulessym} \to \Model \\
    \predictDLsym  & : & \Model \to t^a \to \errorrulessym \\
    \midrule
    \diffsym       & : & \pairsym \to \errorrulessym \\
    \ecepsym       & : & \errorrulessym \to e_{\bot} \to e \\
    \trainsym      & : & \datasetsym \to \Model \\
    \predictsym    & : & \Model \to G \to e_{\bot} \to \errorrulessym \\
    \midrule
    \systemsym     & : & G \to \datasetsym \to (e_{\bot} \to e) \\
  \end{array}
  }
  \]
  \lstMakeShortInline[language=Python, mathescape=true]{|}
\end{minipage}
\caption{A high-level API for building a \toolname system that learns parse
error repair function $e_{\bot} \to e$ from a $\datasetsym$ of program pairs.}
\label{fig:api}
\end{figure}


\mypara{Lexical Analysis}
%
\emph{Lexical analysis} (or lexing or tokenization) converts
a sequence of characters into a sequence of tokens comprising
strings with an assigned and thus identified meaning (\eg numbers,
identifiers \etc).
%
Lexing is usually combined with a parser, which together analyze
the syntax of a programming language $L(G)$, defined by the grammar $G$.
%
When a program has a syntax error, the output token sequence of the
lexer is the highest available level of abstraction as, since the
parser fails without producing a parse tree.

\mypara{Token Sequences}
%
Our goal is to parse a \emph{program token sequence} $t^i$, which is a lexed
program \emph{with} parse errors (\ie $t^i \notin L(G)$), and repair it into a
\emph{fixed token sequence} $t^o \in L(G)$ that can be used to return a repaired
program without syntax errors. Let $t^i$ be a sequence $t^i_1, t^i_2, \dots,
t^i_n$ and $t^o$ be the updated sequence $t^o_1, t^o_2, \dots, t^o_i, \dots,
t^o_j, \dots, t^o_k$. The subsequence $t^o_i, \dots, t^o_j$ can either
\emph{replace} a subsequence in $t^i$, it can \emph{be inserted} in $t^i$ or can
be the empty subsequence $\epsilon$ and \emph{delete} a subsequence
in $t^i$ to generate the $t^o$. It can be the whole program, part of it or
multiple parts of it. $t^o$ will finally be a token sequence that can be parsed
by the original language's $L(G)$ parser.

However, programs and hence, $n$ can be large which makes
them unsuitable for training effectively sequence models.
Therefore, our goal is to first generate an \emph{abstracted token sequence}
$t^a$ that removes all irrelevant information from $t^i$
and gives hints for the parse error fix by using the
internal states of an \emph{Earley} parser.


\subsection{Earley Partial Parses}
\label{sec:prog-abstract:partial}

\toolname uses an \emph{Earley parser} to generate the abstracted token
sequence $t^a$ for an input program sequence $t^i$. An Earley parser holds
internally a \emph{chart} data structure, \ie \emph{a list of partial parses}.
Given a production rule $X \rightarrow \alpha \beta$, the notation $X
\rightarrow \alpha \cdot \beta$ represents a condition in which $\alpha$ has
already been parsed and $\beta$ is expected and both are sequences of terminal
and non-terminal symbols (tokens).

Each state is a tuple $(X \rightarrow \alpha \cdot \beta,\ j)$, consisting of
\begin{itemize}
    \item the production rule currently being matched $(X \rightarrow \alpha
    \beta)$
    \item the current position in that production (represented by the dot
    $\cdot$)
    \item the origin position $j$ in the input at which the matching of this
    production began
\end{itemize}

We denote the state set at an input position $k$ as $S(k)$. The parser is seeded
with $S(0)$ consisting of only the top-level rule $S \rightarrow \gamma$. It
then repeatedly executes three operations: \emph{prediction, scanning,} and
\emph{completion}. There exists a \emph{complete parse} if the complete
top-level rule $(S \rightarrow \gamma \cdot, 0)$ is found in $S(n)$, where $n$
the input length. We define a \emph{partial parse} to be any partially completed
rules, \ie if there is $(X \rightarrow \alpha \cdot \beta, i)$ in some state
$S(k)$, where $i < k \leq n$.

\GS{TODO: Updated this section to highlight the iterative nature of the
abstraction algorithm. Check again.}

Let, $t^i_1, t^i_2, \dots, t^i_j, \dots, t^i_k, \dots, t^i_n$ be the
input token sequence $t^i$, where there is a parse error at location $k$ and the
parser has exhausted all possibilities and can not add any more rules in state
$S(k + 1)$, \ie $S(k + 1) = \emptyset$. We want to abstract program subsequences
$t^i_j, \dots, t^i_k$ by getting the longest possible parts of the program $t^i$
that have a partial parse. For example, we start from the beginning of the
program $t^i_1$ by finding the largest $j$ for which there is a rule $(X
\rightarrow \alpha \cdot \beta, 0) \in S(j)$. We use this rule for $X$ to
replace $t^i_1, t^i_2, \dots, t^i_j$ in $t^i$ with $\alpha$, thus getting an
abstracted sequence $t^a$. In the same manner, we use the longest possible
partial parses that we can extract from the chart to abstract $t^i_{j+1}, \dots,
t^i_k$, iteratively, until we reach the parse error at location $k$.

\mypara{Problem: Multiple Partial Parses}
%
However, each of the states $S(j),\ 0 \leq j \leq k$, holds a large number of
partial parses and, thus, our heuristic to choose the longest possible partial
parse to abstract programs may not be able to abstract the token sequence fully
until the error location $k$, or not even until the end location $n$ of the
program that may not have any more parse errors. Additionally, there may be two
or more partial parses in $S(k)$, with different lengths, \eg $\{(X \rightarrow
\alpha \cdot \beta,\ j),\ (X' \rightarrow \alpha' \cdot \beta',\ h)\} \in S(k),\
j \neq h$. We propose selecting the most \emph{probable parse} with the aid of a
PCFG.


\subsection{Probabilistic Context-Free Grammars}
\label{sec:prog-abstract:pcfg}

We learn a PCFG from a large corpus of programs $\List{e}, e \in L(G)$, that
belong to a language $L(G)$, that a grammar $G$ defines, with the
$\learnPCFGsym$ procedure as shown in \autoref{fig:api}.
%
We use the learned PCFG with an augmented Earley parser in
$\partialsym$ to abstract a program $e_{\bot}$ into a abstract
token sequence $t^a$.

\mypara{Probabilistic CFG}
%
A PCFG can be defined similarly to a \emph{context-free grammar} $G \defeq (N,\
\Sigma,\ P,\ S)$ as a quintuple $(N,\ \Sigma,\ P,\ S,\ W)$, where:
\begin{itemize}
    \item $N$ and $\Sigma$ are finite disjoint alphabets of non-terminals and
    terminals, respectively.
    \item $P$ is a finite set of production rules of the form $X \rightarrow
    \alpha$, where $X \in N$ and $\alpha \in (N \cup \Sigma)^{\ast}$.
    \item $S$ is a distinguished start symbol in $N$.
    \item $W$ is a finite set of probabilities $p(X \rightarrow \alpha)$ on
    production rules.
\end{itemize}


Given a dataset of programs $\List{e}, e \in L(G)$ that can be parsed, let
$\ccount(X \rightarrow \alpha)$ be the number of times the production rule $X
\rightarrow \alpha$ has been used to generate a final complete parse, while
parsing $\List{e}$, and $\ccount(X)$ be the number of times the non-terminal $X$
has been seen in the left side of a used production rule. The probability for a
production rule $X \rightarrow \alpha$ is then defined as:

\begin{equation*}
    p(X \rightarrow \alpha) = \frac{\ccount(X \rightarrow \alpha)}{\ccount(X)}
\end{equation*}

% Of course, under this definition we have the constraint that for any $X \in N$:

% \begin{equation*}
%     \sum_{X \rightarrow \alpha \in P, \alpha \in (N \cup \Sigma)^{\ast}}{p(X \rightarrow \alpha)} = 1
% \end{equation*}

$\learnPCFGsym$ invokes an instrumented Earley parser to calculate all the
values $\ccount(X \rightarrow \alpha), \forall X \rightarrow \alpha: P$ and
$\ccount(X), \forall X: N$. The \emph{instrumented parser} keeps a \emph{global record}
of these values, while parsing the dataset $\List{e}$ of programs.
%
Finally, $\learnPCFGsym$ outputs a PCFG that is based on the original grammar
$G$ that was used to parse the dataset with the learned probabilities $W$.

\subsection{Abstracted Token Sequences}

Given a program $e_{\bot}$ with a parse error and a learned PCFG, $\partialsym$
will generate an abstracted token sequence $t^a$. The PCFG will be used with an
\emph{augmented Earley parser} to disambiguate partial parses and choose one, in
order to produce an abstracted token sequence as described in
\S~\ref{sec:prog-abstract:partial}.

% \mypara{Probabilistic Earley Parsing}
%
We augment Earley states $(X \rightarrow \alpha \cdot \beta,\ j)$ to
$(X \rightarrow \alpha \cdot \beta,\ j,\ p)$, where $p$ is the probability that $X
\rightarrow \alpha \cdot \beta$ is a correct partial parse.
%
When there are two (or more) conflicting partial parses
$\{(X \rightarrow \alpha \cdot \beta,\ j,\ p),\ (X' \rightarrow \alpha' \cdot \beta',\ h,\ p')\} \in S(k)$,
the augmented parser selects the partial parse with the highest probability $max(p,\ p')$.
%
The augmented parser calculates the probability $p$ for a partial parse
$(X \rightarrow \alpha \cdot \beta,\ j,\ p)$ in the state $S(k)$, as the
product $p_1 \cdot p_2 \cdot \dots \cdot p_{k-1}$ of the probabilities
$p_1,\ p_2,\ \dots,\ p_{k-1}$ that are associated with the production
rules $(X_1 \rightarrow \alpha_1 \cdot \beta_1,\ i_1,\ p_1),
(X_2 \rightarrow \alpha_2 \cdot \beta_2,\ i_2,\ p_2), \dots$
that have been used so far to parse the string of tokens $\alpha$.