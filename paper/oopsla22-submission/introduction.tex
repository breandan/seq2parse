
\section{Introduction}
\label{sec:intro}

Parse errors can vex novices who
are learning to program.
%
Traditional error messages
only indicate the first error
or produce messages that
are either incomprehensibly
verbose or not descriptive
enough to help swiftly remedy
the error~\citep{qian2017, VanDerSpek_2005}).
%
When they occur in larger code bases, parse errors
may even trouble experts, and can require a great deal of
effort to fix~\citep{Denny_2012, Ahadi_2018, Kummerfeld2003}.

Owing to their ubiquity and importance, there are \emph{two}
established lines of work on automatically suggesting \emph{repairs}
for parse errors.
%
In the first line, Programming Languages researchers
have investigated \emph{symbolic} approaches starting
with classical parsing algorithms, \eg, LR~\citep{Aho1974}
or Earley~\citep{Earley_1970}.
%
These algorithms can accurately locate syntax errors,
but do not provide \emph{actionable} feedback on how
to fix the error.
%
\citet{Aho_1972} extends these ideas to implement
\emph{error correcting parsers} (EC-Parsers) that
use special error production rules to handle
programs with syntax errors and
synthesize minimal-edit parse error repairs.
%
Sadly, EC-parsers have remained mostly of theoretical
interest, as their running time is cubic in the
program size, and quadratic in the size of the language's
grammar, which has rendered them impractical for
real-world languages~\citep{McLean1996, Rajasekaran2014}.

In the second line, Machine Learning and NLP researchers have
pursued \emph{Deep Neural Network} (\dnn)
approaches using advanced sequence-to-sequence
models \citep{Sutskever_2014, Hardalov_2018}
that use a large corpus of code to predict
the next token (\eg, at a parse error location).
%
Unfortunately, these methods ignore the high-level structure
of the language (or must learn it from vast amounts
of data) and hence, lack accuracy in real-world contexts.
%
For example, state-of-the-art methods such as \citet{Ahmed_2021} parse and
repair only \emph{32\%} of real student code with up to 3 syntax errors
while \citet{Wu2020} repair only \emph{58\%} of syntax errors in a
real-world dataset.

In this paper, we present \toolname, a new approach
to automatically repairing parse errors
based on the following key insight.
%
Symbolic EC-Parsers \citep{Aho_1972} can, in principle,
synthesize repairs, but, in practice, are overwhelmed by
the many error-correction rules that are not \emph{relevant}
to the particular program that requires repair.
%
In contrast, Neural approaches are fooled by the large
space of possible sequence level edits, but can precisely
pinpoint the set of EC-rules that are \emph{relevant}
to a particular program.
%
Thus, \toolname addresses the problem of parse error
repair by a neurosymbolic approach that combines
the complementary strengths of the two lines of work
via the following concrete contributions.

\mypara{1. Motivation}
%
Our first contribution is an empirical analysis of a real-world
dataset of more than a million novice Python programs that shows
that parse errors constitute the majority of novice errors, take a long
time to fix, and that the fixes themselves can often require
multiple edits.
%
This analysis clearly demonstrates that an automated tool that
suggests parse error repairs in a few seconds could greatly
benefit many novices (\S~\ref{sec:error-analysis}).

\mypara{2. Implementation}
%
Our second contribution is the design and implementation
of \toolname, which exploits the insight above to
efficiently and accurately suggest repairs in a
neurosymbolic fashion:
%
(1) train sequence classifiers to predict the \emph{relevant}
EC-rules for a given program (\S~\ref{sec:seq-classifiers}), and then
%
(2) use the predicted rules to synthesize repairs
via EC-Parsing (\S~\ref{sec:whole-system}).

\mypara{3. Abstraction}
%
Predicting the rules is challenging.
Standard NLP token-sequence based
methods are confused by long
trailing contexts that are
independent of the parse error.
%
This confusion yields to inaccurate
classifiers that predict irrelevant
rules yielding woefully low repair rates.
%
Our second key insight eliminates neural confusion
via a symbolic intervention: we show how to use
Probabilistic Context Free Grammars (PCFGs)
to \emph{abstract} long low-level token
sequences so that the irrelevant trailing
context is compressed into single non-terminals,
yielding compressed abstract token sequences
that can be accurately understood by \dnn{}s
(\S~\ref{sec:prog-abstract}).

\mypara{4. Evaluation}
%
Our final contribution is an evaluation of \toolname
using a dataset of more than 1,100,000 Python programs
that demonstrates its benefits in three ways.
%
First, we show its \emph{accuracy}: \toolname correctly predicts
the right set of error rules $81\%$ of the time when considering the top $20$
rules and can parse $94\%$ of our tests within $2.1$ seconds with these
predictions, a significant improvement over prior methods
which were stuck below a $60\%$ repair rate.
%
Second, we demonstrate its \emph{efficiency}: \toolname parses and repairs
erroneous programs within $20$ seconds $83\%$ of the time, while also generating
\emph{the user fix in almost 1 out 3 of the cases}.
%
Finally, we measure the \emph{quality} of the generated repairs via a human
study with 39 participants and show that humans perceive both \toolname's edit
locations and final repair quality to be useful and helpful, in a
statistically-significant manner, even when not equivalent to the user's fix
(\S~\ref{sec:eval}).

% The rest of the paper is organized as follows: \autoref{sec:error-analysis}
% shows the importance of syntax errors in novice software
% development.
% In \autoref{sec:overview} we present a high-level overview, while
% \autoref{sec:prog-abstract}, \autoref{sec:seq-classifiers} and
% \autoref{sec:whole-system} describe the algorithmic implementation of
% \toolname. Finally, \autoref{sec:eval} presents quantitative and
% qualitative results and \autoref{sec:related-work} places our work in
% context.
