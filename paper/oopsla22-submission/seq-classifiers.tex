\section{Training Sequence Classifiers}
\label{sec:seq-classifiers}

Our next task is to \emph{train} a model that can predict which error production
rules a program (with syntax errors) needs to get parsed accordingly with the
given grammar, given its (abstracted) program token sequence.
%
We do so by defining a function $\predictDLsym$ which takes as input a
\emph{pre-trained Deep Neural Network (\dnn)} model $\Model$ and an abstracted
token sequence $t^a$.
%
It returns as output a \emph{fixed set of error production rules}
$\errorrulessym$ which can be used along with an error correcting parser (ECEP)
(\autoref{sec:whole-system}) to successfully parse the program.
%
We acquire the pre-trained $\Model$ using the $\trainDLsym$ method.
$\trainDLsym$ gets a dataset of token sequences $t^a$ and the \emph{exact small
set} of error production rules that the ECEP needs to generate the intended user
parse, \ie $\List{t^a \times \errorrulessym}$.

Next, we provide some necessary background about \dnn{}s and the
\emph{Sequence-to-sequence (seq2seq) models} and describe how we incorporate
them in our problem, framing it as a instance of \emph{multi-class
classification}. We leave the high level details of acquiring the dataset of
labeled token sequences and using the predictor for new erroneous programs for
\autoref{sec:whole-system}.


\subsection{Sequence Models}
\label{sec:seq-classifiers:seq-models}

\mypara{Background: Sequence-To-Sequence Models}
Sequence-to-sequence (seq2seq) models aim to transform input sequences of one
domain into sequences of another domain \citep{Sutskever_2014}. In the general
case these models consist of two major layers, an \emph{encoder} and a
\emph{decoder}. The encoder transforms an input token sequence $x_1, x_2, \dots,
x_n$ into a \emph{abstract vector} $V \in \R^k$ that captures all the essence
and context of the input sequence. This vector doesn't necessarily have some
physical meaning and is just an internal representation of the input sequence
into a higher dimensional space. The abstract vector is then given as an input
to the decoder, which in turn transforms it into an output sequence $y_1, y_2,
\dots, y_n$.

\begin{align}
    h_t &= f(W^{h\times x} x_t + W^{h\times h} h_{t-1}) \label{eq:1} \\
    y_t &= g(W^{y\times h} h_t) \label{eq:2}
\end{align}

The simplest approach historically uses a Recurrent Neural Network (RNN)
\citep{Rumelhart1986, Werbos1990, Hochreiter_1997}, which is a natural next step
from the classic neural networks. Each RNN unit however operates on each input
token $x_t$ separately. It keeps an internal \emph{hidden state} $h_t$ that is
calculated as in \autoref{eq:1}, as a function of the input token $x_t$ and the
previous hidden state $h_{t-1}$. The weight matrices $W^{h\times x}$ and
$W^{h\times h}$ parametrize the input-to-hidden and the hidden-to-hidden
connections respectively. The function $f$ is a \emph{non-linear activation}
function such as \emph{tanh, sigmoid, ReLU}. The output $y_t$ at position $t$ is
calculated as the product of the current hidden state $h_t$ and the weight
matrix $W^{y\times h}$ (\autoref{eq:2}). The activation function $g$ is usually
chosen as the standard \emph{softmax} function, which, for an output vector $y =
(y_1, \dots, y_N) \in \R^{N}$, is defined as:
\[ \sigma(y)_i = \frac{e^{y_i}}{\sum_{j=1}^{N} e^{y_j}},\ for\ i = 1, \dots, N.
\] Finally, the loss function at all steps of the RNN is typically calculated
as the sum of the cross-entropy loss of each step.

The recurrent models such as the RNN, have to generate the sequence of hidden
states $h_t$, as a function of the previous hidden state $h_{t-1}$ and the input
for position $t$. This sequential generation of hidden states limits
parallelization within training examples and causes training problems when
dealing with longer sequences, as memory constraints limit batching across
examples. However, the latest state-of-the-art approach for the seq2seq
architecture is the newer \emph{Transformer} model \citep{Vaswani_2017} that
replaces the classic RNN unit in the encoder and decoder and solves these
problems due to its \emph{attention mechanisms}.

\mypara{Transformers}
Attention mechanisms have been of interest lately \citep{Bahdanau2015, Kim2017,
Vaswani_2017} mainly due to their ability to detect dependencies in the input or
output sequences regardless the distance of the tokens. The Transformer is a
model architecture that deviates from the recurrent pattern (\eg RNNs) and is
solely relying on attention mechanisms. The nature of this architecture makes
the Transformer significantly more easy to parallelize and has as a result a
higher quality of predictions and sequence translations after only being trained
for a small amount of time.

The Transformer follows this overall architecture using stacked self-attention
and point-wise, fully connected layers for both the encoder and decoder, shown
in the left and right halves of Figure 1, respectively.

Scaled Dot-Product Attention, computes the attention function on a set of
queries simultaneously, packed together into a matrix.

Multi-head attention, allows the model to jointly attend to information from
different representation subspaces at different positions.

A self-attention layer connects all positions with a constant number of
sequentially executed operations, whereas a recurrent layer requires O(n)
sequential operations.

In terms of computational complexity, self-attention layers are faster than
recurrent layers when the sequence length is smaller than the representation
dimensionality, which is often the case with machine translations.

\subsection{Transformer Classifier}
\label{sec:seq-classifiers:location-rank}

Our goal with the $\trainsym$ function is to train two separate
\emph{classifiers} given a training set $\List{\datasym \times
\labelsym{\ConfBin}}$ of labeled examples. $\Model$ predicts error locations and
$\Model$ predicts fix templates for a new input program $\pbad$. Critically, we
require that the error localization classifier output a \emph{confidence score}
$\Conf$ that represents the probability that a subexpression is the error that
needs to be fixed. We also require that the fix template classifier output a
confidence score $\Conf$ for each fix template that measures
how sure the classifier is that the template can be used to repair the
associated location of the input program $\pbad$.

We consider a standard learning algorithm to generate our models: \emph{neural
networks}. A thorough introduction to neural networks is
beyond the scope of this work~\citep{Hastie2009-bn,Nielsen2015-pu}.

\mypara{Neural Networks}
The model that we use is a type of neural network called a \emph{multi-layer
perceptron}. A multi-layer perceptron can be represented as a directed acyclic
graph whose nodes are arranged in layers that are fully connected by weighted
edges. The first layer corresponds to the input features, and the final to the
output. The output of an internal node is the sum of the weighted
outputs of the previous layer passed to a non-linear function, called the
activation function. The number of layers,
the number of nodes per layer, and the connections between layers constitute the
\emph{architecture} of a neural network. In this work, we use relatively
\emph{deep neural networks} (\dnn). We can train a \dnn $\Model$ as a
binary classifier, which will predict whether a location in a program $\pbad$
has to be fixed or not.

\mypara{Multi-class \dnn{}s}
While the above model is enough for error localization, in the case of template
prediction we have to select from more than two \emph{classes}. We again use a
\dnn for our template prediction $\Model$, but we adjust the output layer to
have $N$ nodes for the $N$ chosen template-classes. For multi-class
classification problems solved with neural networks, usually a \emph{softmax}
function is used at output layer \citep{Goodfellow-et-al-2016,Bishop-book-2006}.
Softmax assigns probabilities to each class that must add up to 1.
This additional constraint speeds up training.

\mypara{Training Multi-class Transformer Classifiers}

\mypara{Predicting Error Rules}

Our ultimate goal is to be able to pinpoint what parts of an erroneous program
should be repaired and what fix templates should be used for that purpose.
Therefore, the $\predictsym$ function uses $\ranksym$ to predict all
subexpressions' confidence scores $\Conf$ to be an error location and confidence
scores $\Tmap{\Conf}$ for each fix template. We show here how all the functions
in our high-level API in \autoref{fig:api} are combined to produce a final list
of confidence scores for a new program $p$. \autoref{algo:predict-algo} presents
our high-level $\predictsym$ algorithm.

\mypara{The Prediction Algorithm}
Our algorithm first extracts the machine-learning-amenable dataset $D_{ML}$
from the program pairs dataset $D$. For each program pair in $D$,
\textsc{Extract} returns a mapping from the erroneous program's subexpressions
to features and labels. Then, \textsc{InSlice} keeps only the expressions in
the the type-error slice and evaluates to a list of the respective feature and
label vectors, which is added to the $D_{ML}$ dataset. This dataset
is used by the \textsc{Train} function to generate our predictive $Models$, \ie
$\Model$ and $\Model$.

At this point we want to generate a $\predictorsym$ for a new unknown program
$p$. We perform feature extraction for $p$ with \textsc{Extract}, and use
\textsc{InSlice} to restrict to expressions in $p$'s type-error slice.
The result is given by $Data(p)$.

\textsc{Rank} is then applied to all subexpressions produced by $Data(p)$ with
$\textsc{Map}$, which will create a mapping of the type
$\Emap{(\labelsym{\Conf})}$ associating expressions with confidence scores. We
apply \textsc{Rank} to each feature vector that corresponds to an expression in
the type-error slice of $p$. These vectors are the first elements of $\tilde{p}
\in Data(p)$, which are of type $\datasym \times \labelsym{\ConfBin}$. Finally,
$\predictorsym\ Pr$ is returned, which is used by our synthesis algorithm in
\autoref{sec:synthesis} to correlate subexpressions in $p$ with their confidence
scores.
