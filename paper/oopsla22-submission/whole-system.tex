\section{Building a fast error-correcting parser}
\label{sec:whole-system}

We show how $\systemsym$ uses the abstracted token sequences from
\autoref{sec:prog-abstract} and the trained sequence models from
\autoref{sec:seq-classifiers} to generate an \emph{error-correcting parser}
$(e_{\bot} \to e)$, that takes as input a program $e_{\bot}$ with parse errors,
parsers it, and produces a correct program $e$.


\subsection{Learning Error Production Rules}
\label{sec:whole-system:error-rules}
We implement $\diffsym$, a function that extracts the \emph{token differences}
between an erroneous program $e_{\bot}$ and a fixed program $e$. $\diffsym$
utilizes an \emph{$O(ND)$ difference algorithm} \citep{Myers_1986} that extracts
possible \emph{insertions, deletions,} and \emph{replacements} to the original
program $e_{\bot}$ to produce the fixed program $e$. We can then map these
changes to the relevant error production rules that are needed to repair
$e_{\bot}$.

\mypara{Extracting error rules efficiently from a dataset}
The $\trainDLsym$ method requires a dataset of token sequences $t_a$ that is
annotated with an \emph{exact and small set} of error production rules, \ie
$\List{t_a \times \errorrulessym}$. The straightforward approach is to use the
Error-Correcting Earley parser $\ecepsym$ with all possible error production
rules $\errorrulessym$ as input for each program $e_{\bot}$. Then, when ECEP
returns with a successful parse we can extract from it the rules that where used
to repair and parse the program $e_{\bot}$. This would produce a dataset with
the most accurate error rules as labels. However, this approach, first,
completely ignores the programmer's fix, which might not be the minimum-distance
edit that the ECEP would produce but contains the intended changes, and, second,
can take an unreasonable amount of time to generate for a dataset with millions
of programs, due to the inefficient nature of the ECEP.

Therefore, we suggest using an $O(ND)$ difference algorithm to get a small and
still representative set of error production rules for each program $e$. We
employ this algorithm to find the differences between, again, the input
\emph{program token sequence} $t^i$, which is the lexed program $e_{\bot}$ and
the \emph{fixed token sequence} $t^o$, which is the lexed program $e$. This
algorithm returns changes between the token sequences in the form of
\emph{inserted or deleted tokens}. It is possible that this algorithm returns a
sequence of deletions followed by a sequence of insertions, which can in turn
interpreted as \emph{replacement}. Therefore, we map these three types of
changes to the respective error production rules. Let $t^i$ be a sequence
$t^i_1, t^i_2, \dots, t^i_n$ and $t^o$ be the updated sequence $t^o_1, t^o_2,
\dots, t^o_k$. We map:
\begin{itemize}
    \item an inserted output token $t^o_j$ to a \emph{deletion} error $E_{t^o_j}
    \rightarrow e$.
    \item an deleted input token $t^i_k$ to an \emph{insertion} error $I
    \rightarrow t^i_k$.
    \item a replaced token $t^i_k$ with $t^o_j$ to a \emph{replacement} error
    $E_{t^o_j} \rightarrow t^i_k$.
\end{itemize}

In the case of an insertion error, we include the helper production rules $H
\rightarrow HI\ \vert\ I$. Since $H$ can derive any nonempty string, the
production $E_a \rightarrow Ha$ introduces a sequence of one or more insertion
errors in front of $a$. Therefore, for the case of an insertion error $I
\rightarrow t^i_k$, we also add $E_{t^i_{k+1}} \rightarrow Ht^i_{k+1}$. To place
one or more insertion errors at the end of a sentence we also include the
production $S' \rightarrow SH$.

The above algorithm, so far, adds only the \emph{terminal error productions}. We
have to also include the \emph{non-terminal error productions} that will invoke
the terminal ones. If $X \rightarrow a_0b_0a_1b_1 \dots a_mb_m,\ m \geq 0$, is a
production in $P$ such that $a_i$ is in $N^*$ and $b_i$ is in $\Sigma$, then we
add the error production $X \rightarrow a_0X_{b_0}a_1X_{b_1} \dots a_mX_{b_m},\
m \geq 0$ to $P'$, where each $X_{b_i}$, is either a new non-terminal $E_{b_i}$
that was added with the above algorithm, or just $b_i$ again if it was not
added.

Finally, we further refine the new small set of error productions for each
program $e_{\bot}$ with ECEP, in order to create the final annotated dataset
$\List{t_a \times \errorrulessym}$. The changes that we extracted from the
programmers' fixes might include irrelevant changes to the parse error fix.
Therefore, running ECEP is still essential to annotate each program with the
appropriate error production rules as described before. However, limiting the
error rules to only the extracted small set instead of every possible error
rule, makes the process of generating this dataset a much more tractable and
efficient process.


\subsection{Training the Classifier}
\label{sec:whole-system:training-classifier}

\begin{algorithm}[t]
    \caption{Training $\systemsym$'s model $\Model$}
    \label{algo:training-classifier-algo}
    \renewcommand{\algorithmicrequire}{\textbf{Input:}}
    \renewcommand{\algorithmicensure}{\textbf{Output:}}
    \begin{algorithmic}[1]
    \Require{Probabilistic Grammar $G$, $\datasetsym\ Ds$}
    \Ensure{Classifier $Model$}
    \Procedure{Train}{$G,\,Ds$}
    \State $D_{ML} \gets \emptyset$
    \ForAll{$\pbad \times \pfix \in Ds$}
      \State $t^a \gets$ \Call{PartialParse}{$G,\,\pbad$}
      \State $rules \gets$ \Call{DiffRules}{$\pbad \times \pfix$}
      \State $D_{ML} \gets D_{ML}\,\cup\ (t_a \times rules)$
    \EndFor
    \State $Model \gets$ \Call{TrainDL}{$D_{ML}$}
    \State \Return{$Model$}
    \EndProcedure
    \end{algorithmic}
\end{algorithm}


\mypara{Training the Transformer Classifier}
We describe in \autoref{algo:training-classifier-algo} how we use a
(probabilistic) grammar $G$ and a dataset $Ds$ to train a Transformer classifier
$Model$ that can be used to predict error production rules for erroneous
programs $\pbad$. We define the $\trainsym$ procedure in
\autoref{algo:training-classifier-algo} that is responsible for extracting a
machine-learning-appropriate dataset $D_{ML}$ in order to train the Transformer
classifier with $\trainDLsym$.

The dataset $D_{ML}$ starts as an empty set. Then we perform the following steps
for each pair $\pbad \times \pfix$: First, we employ $\partialsym$ with input
grammar $G$ and the erroneous program $\pbad$ to extract the abstracted token
sequence $t^a$. Second, we use the token difference algorithm $\diffsym$ to
extract the specific error production $rules$ that are needed to fix $\pbad$ and
will be used as the training labels of the $D_{ML}$ dataset. Finally, the
abstract sequence $t^a$ is annotated with the label $rules$ and is added to
$D_{ML}$.

A trained Transformer classifier is then learned by $\trainDLsym$ that uses the
newly extracted dataset $D_{ML}$. The learned classifier $Model$ is returned in
order to be used later on to fix new erroneous programs. This whole procedure
happens offline and therefore it won't affect the performance of a final repair.

\begin{algorithm}[t]
    \caption{Predicting Templates Algorithm}
    \label{algo:predict-algo}
    \renewcommand{\algorithmicrequire}{\textbf{Input:}}
    \renewcommand{\algorithmicensure}{\textbf{Output:}}
    \begin{algorithmic}[1]
    \Require{Feature Extraction Functions $F$, Fix Templates $Ts$, Program Pair Dataset $D$}
    \Ensure{Predictor $Pr$}
    \Procedure{Predict}{$F,\,Ts,\,D$}
    \State $D_{ML} \gets \emptyset$
    \ForAll{$\pbad \times \pfix \in D$}
      \State $d \gets$ \Call{Extract}{$F,\,Ts,\,\pbad \times \pfix$}
      \State $D_{ML} \gets D_{ML}\,\cup$ \Call{InSlice}{$\pbad,\,d$}
    \EndFor
    \State $Models \gets$ \Call{Train}{$D_{ML}$}
    \State $Data \gets \lambda p.$ \Call{InSlice}{$p,$ \Call{Extract}{$F,\,Ts,\,p \times p$}}
    \State $Pr \gets \lambda p.$ \Call{Map}{$\lambda \tilde{p}.$ \Call{Rank}{$Models,\,\tilde{p}[0]$}, $Data(p)$}
    \State \Return{$Pr$}
    \EndProcedure
    \end{algorithmic}
\end{algorithm}


\mypara{Predicting Error Rules}
Having trained a Transformer classifier $Model$, we can now predict error rules
$Rls$, that will be used by an ECEP, by using the $\predictsym$ procedure
defined in \autoref{algo:predict-algo}. $\predictsym$ uses the same input
grammar $G$ to generate an abstracted token sequence $t^a$ for the program $P$
with the $\partialsym$ procedure. Finally, the $\predictDLsym$ procedure can
predict a small set of error production rules $Rls$ for the abstracted sequence
$t^a$ given the pre-trained $Model$.

\subsection{Generating an error-correcting parser}
\label{sec:whole-system:building-ecp}

\begin{algorithm}[t]
    \captionsetup{font=small}
    \caption{Generating the final ECEP}
    \label{algo:ecep-algo}
    \renewcommand{\algorithmicrequire}{\textbf{Input:}}
    \renewcommand{\algorithmicensure}{\textbf{Output:}}
    \begin{algorithmic}[1]
    \Require{Grammar $G$, $\datasetsym\ Ds$}
    \Ensure{Error Correcting Parser $Prs$}
    \Procedure{Seq2Parse}{$G,\,Ds$}
    \State $ps \gets$ \Call{Map}{$\lambda.p \to$ \Call{Snd}{$p$}$,\,Ds$}
    \State $PCFG \gets$ \Call{LearnPCFG}{$G,\,ps$}
    \State $Model \gets$ \Call{Train}{$PCFG,\,Ds$}
    \State \textsc{ERulePredictor} $\gets$ \Call{Predict}{$Model,\,PCFG$}
    \State $Prs \gets (\lambda.\pbad \to$ \Call{EarleyECParse}{\Call{ERulePredictor}{$\pbad$}$,\,\pbad$}$)$
    \State \Return{$Prs$}
    \EndProcedure
    \end{algorithmic}
\end{algorithm}


We now describe our approach for $\systemsym$ in \autoref{algo:ecep-algo}. This
is the high-level algorithm that combines everything that we described so far in
the last three sections.

In the first offline step, q$\systemsym$ first extracts the fixed programs $ps$
from the dataset $Ds$ to learn a probabilistic context-free grammar $PCFG$ for
the input grammar $G$. Then it $\trainsym$s the $Model$ to predict error
production rules. Then we define an error rule predictor function
\textsc{ERulePredictor} using the $\predictsym$ from \autoref{algo:predict-algo}
withe the pre-trained $Model$ and grammar $PCFG$. Finally, we return the error
correcting parser $Prs$, which we define as a function here that takes as input
an erroneous program $\pbad$, used \textsc{ERulePredictor} to get the set of
error rules needed by $\ecepsym$ to parse and repair the program $\pbad$.
