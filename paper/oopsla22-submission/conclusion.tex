\section{Conclusion}
\label{sec:conclusion}

We have presented \emph{neurosymbolic parse program repair}, a new neurosymbolic
approach to automatically repair parse errors.
%
Our approach is to use a dataset of ill-parsed programs and their fixed versions
to train a Transformer classifier (neural component) which allows us to
accurately predict EC-rules for new programs with syntax errors. In order to
make accurate predictions, we abstract the low-level program token sequences
using partial parses and probabilistic grammars. A small set of predicted
EC-rules is finally used with an ECE-Parser (symbolic component) to parse and
repair new ill-parsed programs in a tractable and precise manner.

We have implemented our approach in \toolname, and demonstrated, using a corpus
of 1,100,000 ill-parsed \python programs drawn from two years of data from an
online web-based educational compiler, that \toolname makes accurate EC-rule
predictions 81\% of the time when considering the top 20 EC-rules, and that the
predicted EC-rules let us parse and repair over 94\% of the test set in 2.1 sec
median parse time, while generating the user fix in almost 1 out of 3 cases.
%
Finally, we conducted a user study with 39 participants which showed that
\toolname's edit locations and repairs are useful and helpful, even when they
are not equivalent to the user's fix.
