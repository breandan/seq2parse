\begin{figure}
    \centering
    \begin{tikzpicture}
    [font=\small, edge from parent,
    every node/.style={top color=white, bottom color=black!15,
    circle, minimum size=9mm, draw=black!75,
    thick, drop shadow, align=center},
    edge from parent/.style={draw=black!75,thick},
    level distance=2cm]
    \node (func) {Funcdef}
        child {
            node (def) {\texttt{\bfseries def}}
        }
        child {
            node (name1) {\texttt{Name}}
        }
        child {
            node (params) {\texttt{Params}}
            child {
                node (openParen) {\texttt{\bfseries (}}
            }
            child {
                node (name2) {\texttt{Name}}
            }
            child {
                node (closeParen) {\texttt{\bfseries )}}
            }
        }
        child {
            node (colon) {\texttt{\bfseries :}}
        }
        child {
            node (suite) {\texttt{Suite}}
            child {
                node (etc) {\texttt{...}}
            }
            child {
                node (ReturnStmt) {\texttt{ReturnStmt}}
                % node (ReturnStmt) {\texttt{\begin{tabular}{c}
                %                                 Return\\
                %                                 Stmt
                %                             \end{tabular}}}
                child {
                    node (return) {\texttt{\bfseries return}}
                }
                child {
                    node (ArgList) {\texttt{ArgList}}
                    child {
                        node (ArithExpr1) {\texttt{ArithExpr}}
                        child {
                            node (ArithExpr2) {\texttt{Literal}}
                        }
                        child {
                            node (plus) {\texttt{\bfseries +}}
                        }
                    }
                }
            }
        };
    \end{tikzpicture}
    \caption{The partial parse tree generated for the example at \autoref{fig:foo-prog}}
    \label{fig:partial-parse-tree}
\end{figure}
