\section{Related Work}
\label{sec:related-work}

There is a vast literature on automatically repairing or patching programs:
we focus on the most closely related work on providing feedback for novice errors.


\mypara{Transformer-based models in Software Engineering}
%
\citep{Rahmani2021, Verbruggen2021} has used pre-trained auto-regressive
transformer models \ie GPT-3 \citep{?} in association with pre-existing program
synthesis techniques. Similar to our work here, this recent work uses
established pre-existing algorithms from the NLP and PL research areas. However,
\citep{Rahmani2021, Verbruggen2021} use pretrained models to acquire semantic
power over smaller subproblems that can't be solved with the syntactic power of
classic program synthesis algorithms. In our work, we train our own
Transformer-based model to augment a classic parsing algorithm in a similar way,
providing though more focused prior knowledge than using an NLP pretrained
model.

% \mypara{Example-Based Feedback}
% %
% Recent work has looked at providing \emph{counterexamples} that show how a
% program went wrong, for type errors \cite{Seidel2016-ul} or for general
% correctness properties where the generated inputs show divergence from a
% reference implementation or other correctness oracle~\cite{Song_2019}. In
% contrast, we provide feedback on how to fix the error.

% \mypara{Fault Localization} Several authors have studied the problem of
% \emph{fault localization}, \ie winnowing down the set of locations that are
% relevant for the error, often using slicing
% \citep{Wand1986-nw,Haack2003-vc,Tip2001-qp,Rahli2015-tt}, counterfactual typing
% \citep{Chen2014-gd} or bayesian methods \citep{Zhang2014-lv}.
% %
% \textsc{Nate}~\citep{Seidel:2017} introduced the BOAT representation,
% and showed it could be used for accurate localization.
% %
% We aim to go beyond localization, into suggesting concrete \emph{changes} that
% novices can make to understand and fix the problem.

% \mypara{Repair-model based feedback}
% %
% \textsc{Seminal} \citep{Lerner2007-dt} \emph{enumerates} minimal fixes using an
% expert-guided heuristic search.
% %
% The above approach is generalized to general correctness properties by
% \cite{singh2013} which additionally performs a \emph{symbolic} search using a
% set of expert provided \emph{sketches} that represent possible repairs.
% %
% In contrast, \toolname learns a template of repairs from a corpus yielding
% higher quality feedback (\S~\ref{sec:eval}).

% \mypara{Corpus-based feedback}
% %
% \textsc{Clara} \citep{Gulwani_2018} uses code and execution traces to match a
% given incorrect program with a ``nearby'' correct solution obtained by
% clustering all the correct answers for a particular task. The matched
% representative is used to extract repair expressions.
% %
% Similarly, \textsc{Sarfgen} \citep{Wang_2018} focuses on structural and
% control-flow similarity of programs to produce repairs, by using AST vector
% embeddings to calculate distance metrics (to ``nearby'' correct
% programs) more robustly.
% %
% \textsc{Clara} and \textsc{Sarfgen} are data-driven, but both assume
% there is a ``close'' correct sample in the corpus.
% %
% In contrast, \toolname has a more general philosophy that \emph{similar errors
% have similar repairs}: we extract generic fix templates that can be applied to
% arbitrary programs whose errors (BOAT vectors) are similar.
% %
% The \textsc{Tracer} system \cite{TRACER2018} is closest in philosophy to ours,
% except that it focuses on single-line compilation errors for C programs, where
% it shows that NLP-based methods like sequence-to-sequence predicting DNNs can
% effectively suggest repairs, %\eg \verb+scanf("%d", a)+ should be converted to
% %\verb+scanf("%d", %a)+
% but this does not scale up to fixing general type errors.
% %
% We have found that \ocaml's relatively simple
% \emph{syntactic} structure but rich \emph{type}
% structure make token-level seq-to-seq methods quite imprecise
% (\eg \emph{deleting} offending statements suffices to ``repair'' C but
% yields ill-typed \ocaml) necessitating \toolname's higher-level semantic
% features and (learned) repair templates.


%
% \textsc{Hercules} \citep{Saha_2019} uses version history to
% repair multiple program locations by using fault localization
% to rank error locations and then generating repairs for each
% location.
% %
% However, it utilizes version history of a big codebase to find
% such repairs, and uses that history to produce repairs that
% modifies multiple program locations.
% %
% \toolname attempts a different approach of multiple location
% repairing, by considering each candidate error location as independent from one
% another.
%
%% While \textsc{Sarfgen} is still a data-driven approach it tries to
%% search for matching programs on the fly, using \toolname's approach can be similarly
%% seen as a per AST node embedding, which however we use in advance for training
%% predictive models, thus mitigating any extra cost on runtime.
%%
%% that can be used to produce repairs. Despite the apparent
%% similarities, \toolname partitions similar fix patterns found in our dataset,
%% while \textsc{Clara} clusters whole student solutions.
%%
%% Additionally, \textsc{Clara} assumes that there is always
%% a matching student solution in the dataset, while \toolname
%% extracts generic fix templates can be applied to arbitrary
%% programs.
%%
%% \textsc{Clara} also scales poorly in matching programs due
%% to the use of Integer Linear Programming, while \toolname's precomputation of
%% fix template models makes final repairs more robust.
